\documentclass[11pt,a4paper]{article}

% Packages
\usepackage[utf8]{inputenc}
\usepackage[T1]{fontenc}
% Computer Modern (default LaTeX font) - no package needed
\usepackage{graphicx}
\usepackage{booktabs}
\usepackage{hyperref}
\usepackage{amsmath}
\usepackage{natbib}
\usepackage[margin=1in]{geometry}
\usepackage{caption}
\usepackage{xcolor}

% Hyperref setup
\hypersetup{
    colorlinks=true,
    linkcolor=blue,
    citecolor=blue,
    urlcolor=blue
}

% Placeholder command
\newcommand{\placeholder}[1]{\textcolor{gray}{\textit{[#1]}}}

\title{The Advice Gap: Gender Disparities in Online Relationship Advice Communities}

\author{
    Henry Stanley\\
    \texttt{henry@henrystanley.com}
}

\date{}

\begin{document}

\maketitle

\begin{abstract}
\placeholder{~150-250 words. Key points to cover:}
\begin{itemize}
    \item Research question: Do men and women receive different advice in online communities?
    \item Method: 6,080 advice comments from 591 Ask Metafilter posts, LLM classification
    \item Main finding: Men receive 2.86x more critical advice (37.9\% vs 17.6\%, $\chi^2 = 282.14$, $p < 0.0001$)
    \item Robustness: Effect persists after controlling for severity, fault, problem type
    \item Validation: 96\% agreement with human judgment on advice direction
    \item Implications: Help-seeking behavior, platform design
\end{itemize}
\end{abstract}

\section{Introduction}

\placeholder{~500 words. Key points to cover:}
\begin{itemize}
    \item Online advice communities are widely used for personal problems
    \item Prior work on gender bias in online spaces (Wikipedia, Stack Overflow, harassment)
    \item Gap: Little research on whether \emph{advice content} differs by recipient gender
    \item This study: Examine advice direction (supportive vs critical) and tone by poster gender
    \item Research questions:
    \begin{enumerate}
        \item Do men/women receive different proportions of critical vs supportive advice?
        \item Do tone labels differ by gender?
        \item Do differences persist after controlling for confounds?
    \end{enumerate}
\end{itemize}

\section{Related Work}

\subsection{Gender Bias in Online Communities}

\placeholder{~200 words. Key citations:}
\begin{itemize}
    \item Wikipedia gender gap \citep{lam2011wp, hill2013wikipedia}
    \item Stack Overflow participation \citep{ford2016paradise}
    \item Online harassment \citep{duggan2017, pew2017}
    \item Feedback differences in professional contexts \citep{correll2016research}
\end{itemize}

\subsection{Advice-Giving Dynamics}

\placeholder{~200 words. Key citations:}
\begin{itemize}
    \item Supportive vs challenging advice \citep{goldsmith2004communicating}
    \item Men and help-seeking, "tough love" norms \citep{addis2003men}
\end{itemize}

\subsection{LLM-Based Content Analysis}

\placeholder{~200 words. Key citations:}
\begin{itemize}
    \item LLMs for content analysis at scale \citep{ziems2024can}
    \item Validation requirements \citep{gilardi2023chatgpt}
    \item Performance on classification tasks \citep{tornberg2023chatgpt}
\end{itemize}

\section{Data and Methods}

\subsection{Data Collection}

\placeholder{~200 words. Key facts:}
\begin{itemize}
    \item Source: Ask Metafilter, "relationships" tag
    \item Known for active moderation, thoughtful responses
    \item Dataset: 591 posts, 7,091 comments (6,080 with advice)
    \item Gender distribution: 1,716 comments on male posts, 4,364 on female posts
    \item Imbalance reflects community composition
\end{itemize}

\subsection{Classification Framework}

\placeholder{~300 words. Describe classification scheme:}

\textbf{Post-level variables:}
\begin{itemize}
    \item Poster gender (from explicit mentions, e.g., "I [30M]...")
    \item Situation severity: low / medium / high
    \item OP fault: none / some / substantial / unclear
    \item Problem category (communication, trust, boundaries, etc.)
\end{itemize}

\textbf{Comment-level variables:}
\begin{itemize}
    \item Is advice: boolean
    \item Advice direction: supportive / critical / neutral / mixed
    \item Tone labels (12 total):
    \begin{itemize}
        \item Positive: gentle, empathetic, constructive, understanding, encouraging, supportive
        \item Negative: harsh, judgmental, blaming, dismissive, condescending, hostile
    \end{itemize}
\end{itemize}

\subsection{LLM Classification}

\placeholder{~200 words. Key points:}
\begin{itemize}
    \item Model: Claude Haiku 4.5
    \item Iteratively refined prompts
    \item Conservative criteria for negative tones (require clear evidence)
    \item Example: "judgmental" = explicitly condemning OP as bad person, not just pointing out mistakes
\end{itemize}

\subsection{Validation}

\placeholder{~150 words. Key points:}
\begin{itemize}
    \item Human spot-checking of 51 random comments
    \item Advice direction: 96\% agreement
    \item Tone labels: ~57\% agreement (more subjective)
    \item Re-classified with conservative criteria
    \item Primary analyses focus on high-reliability advice direction
\end{itemize}

\subsection{Statistical Methods}

\placeholder{~100 words. Methods used:}
\begin{itemize}
    \item Proportions by gender
    \item Odds ratios with 95\% CI
    \item Chi-square tests for independence
    \item Stratified analysis by severity, fault, category
    \item Two-tailed tests, $\alpha = 0.05$
\end{itemize}

\section{Results}

\subsection{Dataset Characteristics}

\placeholder{~100 words. Interpret the table below.}

\begin{table}[h]
\centering
\caption{Post Characteristics by Poster Gender}
\label{tab:characteristics}
\begin{tabular}{lcccc}
\toprule
Variable & Male (n=199) & Female (n=392) & $\chi^2$ & $p$ \\
\midrule
\textbf{Severity} & & & 2.31 & 0.315 \\
\quad Low & 18.1\% & 21.4\% & & \\
\quad Medium & 52.3\% & 48.7\% & & \\
\quad High & 29.6\% & 29.9\% & & \\
\midrule
\textbf{OP Fault} & & & 3.87 & 0.276 \\
\quad None & 31.2\% & 35.7\% & & \\
\quad Some & 42.7\% & 38.5\% & & \\
\quad Substantial & 15.6\% & 17.1\% & & \\
\quad Unclear & 10.5\% & 8.7\% & & \\
\bottomrule
\end{tabular}
\end{table}

\placeholder{Key point: Severity and fault distributions do NOT differ by gender — men and women post about comparable situations.}

\subsection{Primary Finding: Advice Direction}

\placeholder{~150 words. Interpret the table below.}

\begin{table}[h]
\centering
\caption{Advice Direction by Poster Gender}
\label{tab:direction}
\begin{tabular}{lccc}
\toprule
Advice Direction & Male & Female & Difference \\
\midrule
Critical of OP & 37.9\% & 17.6\% & +20.3 pp \\
Supportive of OP & 25.3\% & 45.3\% & $-$20.0 pp \\
Neutral & 24.1\% & 26.8\% & $-$2.7 pp \\
Mixed & 12.7\% & 10.3\% & +2.4 pp \\
\bottomrule
\end{tabular}
\end{table}

\placeholder{Key statistics:}
\begin{itemize}
    \item $\chi^2 = 282.14$, $p < 0.0001$
    \item Odds ratio for critical advice: \textbf{2.86} (95\% CI: 2.50--3.27)
    \item Odds ratio for supportive advice: 0.41 (95\% CI: 0.36--0.47)
\end{itemize}

\subsection{Tone Analysis}

\placeholder{~200 words. Interpret the table below.}

\begin{table}[h]
\centering
\caption{Tone Labels by Poster Gender}
\label{tab:tones}
\begin{tabular}{lccccc}
\toprule
Tone & Male & Female & Diff & $\chi^2$ & $p$ \\
\midrule
\multicolumn{6}{l}{\textbf{Positive tones}} \\
Understanding & 64.9\% & 72.9\% & $-$8.0 pp & 38.7 & $<$0.0001 \\
Empathetic & 54.8\% & 65.5\% & $-$10.7 pp & 60.2 & $<$0.0001 \\
Constructive & 47.3\% & 49.1\% & $-$1.8 pp & 1.7 & 0.19 \\
Supportive & 25.3\% & 45.3\% & $-$20.0 pp & 218.4 & $<$0.0001 \\
Encouraging & 23.7\% & 34.5\% & $-$10.7 pp & 70.1 & $<$0.0001 \\
Gentle & 12.4\% & 18.2\% & $-$5.8 pp & 32.1 & $<$0.0001 \\
\midrule
\multicolumn{6}{l}{\textbf{Negative tones}} \\
Judgmental & 11.7\% & 4.0\% & +7.7 pp & 125.8 & $<$0.0001 \\
Blaming & 9.2\% & 2.2\% & +7.0 pp & 145.3 & $<$0.0001 \\
Harsh & 6.1\% & 3.0\% & +3.1 pp & 30.8 & $<$0.0001 \\
Condescending & 4.5\% & 1.9\% & +2.7 pp & 31.6 & $<$0.0001 \\
Hostile & 2.1\% & 0.3\% & +1.8 pp & 42.9 & $<$0.0001 \\
Dismissive & 3.8\% & 3.2\% & +0.6 pp & 1.3 & 0.25 \\
\bottomrule
\end{tabular}
\end{table}

\placeholder{Key points: Men receive more negative tones (judgmental, blaming, harsh, condescending, hostile). Women receive more positive tones (understanding, empathetic, supportive, encouraging, gentle). Constructive and dismissive show no significant difference.}

\subsection{Confound Analysis}

\placeholder{~200 words. Interpret the stratified tables below.}

\begin{table}[h]
\centering
\caption{Advice Direction by Gender, Stratified by Situation Severity}
\label{tab:severity}
\begin{tabular}{llccc}
\toprule
Severity & Gender & \% Critical & OR & $p$ \\
\midrule
Low & Male & 29.8\% & 2.41 & $<$0.001 \\
 & Female & 15.0\% & & \\
Medium & Male & 38.4\% & 2.92 & $<$0.0001 \\
 & Female & 17.1\% & & \\
High & Male & 43.2\% & 2.78 & $<$0.0001 \\
 & Female & 21.3\% & & \\
\bottomrule
\end{tabular}
\end{table}

\begin{table}[h]
\centering
\caption{Advice Direction by Gender, Stratified by OP Fault}
\label{tab:fault}
\begin{tabular}{llccc}
\toprule
OP Fault & Gender & \% Critical & OR & $p$ \\
\midrule
None & Male & 28.1\% & 2.53 & $<$0.001 \\
 & Female & 13.1\% & & \\
Some & Male & 41.2\% & 2.71 & $<$0.0001 \\
 & Female & 20.4\% & & \\
Substantial & Male & 52.7\% & 2.34 & $<$0.001 \\
 & Female & 31.8\% & & \\
\bottomrule
\end{tabular}
\end{table}

\placeholder{Key points:}
\begin{itemize}
    \item Effect persists across ALL severity levels (OR 2.4--2.9)
    \item Effect persists across ALL fault levels (OR 2.3--2.7)
    \item Notable: Men with NO fault (28.1\%) receive more critical advice than women with SOME fault (20.4\%)
\end{itemize}

\subsection{Sensitivity Analysis}

\placeholder{~100 words. Key points:}
\begin{itemize}
    \item Core finding uses advice direction (96\% human agreement) — robust
    \item Excluding negative tone labels: positive tone differences remain significant
    \item Finding does not depend on subjective tone classifications
\end{itemize}

\section{Discussion}

\subsection{Summary of Findings}

\placeholder{~100 words. Summarize: Men 2.86x more critical advice, persists after controls, women get more supportive/empathetic responses.}

\subsection{Interpretation}

\placeholder{~200 words. Possible mechanisms:}
\begin{itemize}
    \item Commenter stereotypes ("tough love" for men)
    \item Differential accountability standards
    \item Writing style differences (but confound analysis suggests this doesn't fully explain)
    \item Community composition (Ask Metafilter skews female — in-group favoritism?)
\end{itemize}

\subsection{Implications}

\placeholder{~200 words. Discuss implications for:}
\begin{itemize}
    \item Help-seekers (men should contextualize critical feedback)
    \item Communities/platforms (consider bias in advice-giving norms)
    \item Research (LLM methodology for large-scale advice analysis)
\end{itemize}

\subsection{Comparison to Related Work}

\placeholder{~100 words. Compare to:}
\begin{itemize}
    \item Prior work on men receiving less emotional support \citep{addis2003men}
    \item Effect size (OR $\approx$ 3) larger than some professional setting biases
    \item Possible explanation: anonymous online contexts reduce social desirability
\end{itemize}

\section{Limitations}

\placeholder{~300 words. Address:}
\begin{itemize}
    \item Single platform (Ask Metafilter has specific norms/demographics)
    \item LLM classification (validated but imperfect)
    \item Selection effects (can't observe who doesn't post)
    \item Correlation not causation (can't identify mechanism)
    \item Binary gender only (excludes non-binary, undisclosed)
    \item Temporal scope (community norms may change)
\end{itemize}

\section{Conclusion}

\placeholder{~200 words. Key points:}
\begin{itemize}
    \item Evidence of substantial gender disparities in online relationship advice
    \item Men receive critical advice at ~3x the rate of women
    \item Pattern persists across situation types and after controlling for severity/fault
    \item Implications for help-seeking behavior and platform design
    \item Future directions: cross-platform replication, mechanism investigation, interventions
\end{itemize}

\bibliographystyle{plainnat}
\begin{thebibliography}{99}

\bibitem[Addis and Mahalik(2003)]{addis2003men}
Addis, M.~E. and Mahalik, J.~R. (2003).
\newblock Men, masculinity, and the contexts of help seeking.
\newblock \emph{American Psychologist}, 58(1):5--14.

\bibitem[Anthropic(2024)]{anthropic2024}
Anthropic (2024).
\newblock Claude 3.5 Haiku model card.

\bibitem[Correll and Simard(2016)]{correll2016research}
Correll, S.~J. and Simard, C. (2016).
\newblock Research: Vague feedback is holding women back.
\newblock \emph{Harvard Business Review}.

\bibitem[Duggan(2017)]{duggan2017}
Duggan, M. (2017).
\newblock Online harassment 2017.
\newblock Pew Research Center.

\bibitem[Ford et~al.(2016)]{ford2016paradise}
Ford, D., Smith, J., Guo, P.~J., and Parnin, C. (2016).
\newblock Paradise unplugged: Identifying barriers for female participation on Stack Overflow.
\newblock In \emph{Proceedings of the 24th ACM SIGSOFT International Symposium on Foundations of Software Engineering}, pages 846--857.

\bibitem[Gilardi et~al.(2023)]{gilardi2023chatgpt}
Gilardi, F., Alizadeh, M., and Kubli, M. (2023).
\newblock ChatGPT outperforms crowd-workers for text-annotation tasks.
\newblock \emph{arXiv preprint arXiv:2303.15056}.

\bibitem[Goldsmith(2004)]{goldsmith2004communicating}
Goldsmith, D.~J. (2004).
\newblock \emph{Communicating Social Support}.
\newblock Cambridge University Press.

\bibitem[Hill and Shaw(2013)]{hill2013wikipedia}
Hill, B.~M. and Shaw, A. (2013).
\newblock The Wikipedia gender gap revisited: Characterizing survey response bias with propensity score estimation.
\newblock \emph{PloS One}, 8(6):e65782.

\bibitem[Lam et~al.(2011)]{lam2011wp}
Lam, S. T.~K., Uduwage, A., Dong, Z., Sen, S., Musicant, D.~R., Terveen, L., and Riedl, J. (2011).
\newblock WP:Clubhouse? An exploration of Wikipedia's gender imbalance.
\newblock In \emph{Proceedings of the 7th International Symposium on Wikis and Open Collaboration}, pages 1--10.

\bibitem[Pew Research Center(2017)]{pew2017}
Pew Research Center (2017).
\newblock Online harassment 2017.

\bibitem[T{\"o}rnberg(2023)]{tornberg2023chatgpt}
T{\"o}rnberg, P. (2023).
\newblock ChatGPT-4 outperforms experts and crowd workers in annotating political Twitter messages with zero-shot learning.
\newblock \emph{arXiv preprint arXiv:2304.06588}.

\bibitem[Ziems et~al.(2024)]{ziems2024can}
Ziems, C., Held, W., Shaber, O., Lu, J., Levy, M., Lahav, G., et~al. (2024).
\newblock Can large language models transform computational social science?
\newblock \emph{Computational Linguistics}, 50(1):237--291.

\end{thebibliography}

\end{document}
